\documentclass{VUMIFPSkursinis}
\usepackage{algorithmicx}
\usepackage{algorithm}
\usepackage{algpseudocode}
\usepackage{amsfonts}
\usepackage{amsmath}
\usepackage{bm}
\usepackage{caption}
\usepackage{color}
\usepackage{float}
\usepackage{graphicx}
\usepackage{listings}
\usepackage{subfig}
\usepackage{wrapfig}

\usepackage{enumitem}
%PAKEISTA, tarpai tarp sąrašo elementų
\setitemize{noitemsep,topsep=0pt,parsep=0pt,partopsep=0pt}
\setenumerate{noitemsep,topsep=0pt,parsep=0pt,partopsep=0pt}

% Titulinio aprašas
\university{Vilniaus universitetas}
\faculty{Matematikos ir informatikos fakultetas}
\department{Programų sistemų katedra}
\papertype{Kursinis darbas}
\title{Paieškos proceso ir jos rezultatų pateikimo vartotojams panaudojamumas VUL Santaros klinikų puslapyje}
\titleineng{The Usability of the Search Process and Presenting its Results to the User for VUH Santaros klinikos website}
\status{3 kurso 3 grupės studentas}
\author{Tomas Kiziela}
% \secondauthor{Vardonis Pavardonis}   % Pridėti antrą autorių
\supervisor{doc. Kristina Lapin}
\date{Vilnius – \the\year}

% Nustatymai
% \setmainfont{Palemonas}   % Pakeisti teksto šriftą į Palemonas (turi būti įdiegtas sistemoje)
\bibliography{bibliografija}

\begin{document}
	
% PAKEISTA	
\maketitle
\cleardoublepage\pagenumbering{arabic}
\setcounter{page}{2}

%TURINYS
\tableofcontents

\sectionnonum{Įvadas}
%Įvade apibūdinamas darbo tikslas, temos aktualumas ir siekiami rezultatai.
%Darbo įvadas neturi būti dėstymo santrauka. Įvado apimtis 1–2 puslapiai.
Šiame darbe tiriami informacijos paieškos sprendimai leidžiantys palengvinti Vilniaus universiteto ligoninės (VUL) Santaros klinikų internetinio puslapio santa.lt naudojimą ant mobilaus įrenginio. Tyrime bus atsižvelgta į puslapio navigaciją, informacijos paieškos procesą bei gautų rezultatų pateikimą.

Kadangi pacientams internetas lengvai prieinamas, darosi įprasta ieškoti informacijos apie sveikatą ir registruotis internetu\cite{InternetUseByPublicSAEn}\cite{InternetUseByPublicHKEn}. VUL Santaros klinikos yra viena didžiausių Lietuvos ligoninių. Joje dirba virš 5000 darbuotojų, o per metus gydoma apie 1 milijonas ambulatorinių (ateinančių iš namų) pacientų\cite{VulSkApieMusLt}. Taigi santa.lt puslapis yra vienas iš pirmųjų internetinių resursų, kurį pasiekia vartotojai. Puslapyje turėtų būti lengva surasti ieškomą informaciją, nes tai padės pacientams priimti sprendimus apie savo sveikatą.

Tačiau dabartinėje sistemoje vartotojai susiduria su problemomis. Naudojant paieškos sistemą negalima įvesti pilnų žodžių, nes jeigu galūnė bent kažkiek skirsis paieška rezultato negražins. Be to, ieškodamas informacijos apie širdies ligas gauni pilną puslapį padėkų, kurios, nors džiugina, užslepia rezultatus kaip širdies ligų gydymo centro kontaktai. Filtravimas nepadeda, nes gražinti rezultatai yra skirstomi į per plačias kategorijas, kuriose padėkos, naujienos ir puslapio skyriai tokie kaip kontaktai yra vienoje kategorijoje. 

Šio darbo tikslas yra suprojektuoti informacijos architektūrą, kuri leistų pacientams greičiau ir su mažiau paspaudimų rasti aktualią informaciją santa.lt puslapyje. Pacientams aktualu yra registracija pas gydytoją, informacija kaip pasiekti ligoninę ir jos skyrius, informacija apie ligoninės struktūrą. Galutinis darbo rezultatas - puslapio prototipas, kurio informacijos architektūra leidžia greičiau ir su mažiau paspaudimų rasti aktualią informaciją.

Uždaviniai:
\begin{enumerate}
	\item Identifikuoti vartotojų poreikius remiantis literatūros šaltiniais ir internetinių puslapių lankomumo informacija
	\item Išskirti lyginimo kriterijus remiantis literatūros šaltiniais
	\item Atlikti puslapio panaudojamumo analizę remiantis panaudojamumo testavimu
	\item Paruošti sprendimo variantus
	\renewcommand*{\theenumii}{\theenumi.\arabic{enumii}}
	\renewcommand{\labelenumii}{\theenumii}
	\begin{enumerate}
		\item Remiantis panašiais puslapiais ir literatūros šaltiniais išskirti alternatyvius sprendimus
		\item Sukurti sprendimo variantų maketus
		\item Palyginti maketus
		\item Sukurti galutinio sprendimo prototipą
	\end{enumerate}
	\item Išskirti detalius reikalavimus galutinio sprendimo įgyvendinimui
	\item Atlikus literatūros analizę išskirti technologijas padedančias įgyvendinti galutinį sprendimą
\end{enumerate}

\section{Vartotojų poreikių analizė}
Tyrimai nurodo, kad Europoje daugiau nei pusė žmonių bent kartą metuose ieško informacijos apie sveikatą internetu \cite{EuCitizDigHealthEn}, taigi naudotojams aktualu internetinių puslapių panaudojamumas. Nagrinėjant santa.lt puslapio srautą randama, kad naudotojai dažniausiai ateina iš (5,5\%) ir keliauja į (10,4\%) sergu.lt (neįskaitant 19,1\% ateinančių iš google.com ir 21,8\% keliaujančių į google.com)\cite{AlexaSantaEn}. Taigi galima matyti, kad šių puslapių vartotojai iš dalies sutampa ir būtų naudinga atsižvelgti į tai, kokią įtaką daro vienas kitam. Sergu.lt puslapis skirtas išankstinei visų Lietuvos pacientų registracijai internetu. Tai, kad 1 iš 10 santa.lt vartotojų tiesiogiai pereina į sergu.lt puslapį leidžia tikėti, kad vienas iš vartotojų poreikių yra turėti nuorodą į registraciją pas gydytoją. Santa.lt „Kaip mus rasti“ skyrelį vartotojai yra aplankę 1,2 milijonus kartų\cite{VulSkKaipMusRastiLt}, 2 kartus daugiau nei skyrelį „Apie mus“\cite{VulSkApieMusLt}, iš to galima daryti prielaidą apie kitą vartotojų poreikį - sužinoti apie ligoninės klinikų pasiekiamumą.

\section{Sistemų lyginimo kriterijai}
Paieškos sistemos ir informacijos architektūros panaudojamumo vertinimui naudojamos David Travis gairės\cite{SearchGuidelinesEn}\cite{NavigationAndIAGuidelinesEn}.

Panaudojamumo testavimas gali būti atliktas įvairiais metodais. Empiriniai metodai yra plačiausiai naudojami\cite{NielsenUsabilityEn}, tačiau reikalauja daugiau žmonių norint gauti patikimą rezultatą, todėl pasirinktas vienas iš analitinių metodų. Dėl patirties ir laiko stokos pasirinkta naudoti neformaliausią metodą - euristinį vertinimą.

Maketų palyginimui bus naudojamos laiko patvirtintos Jakob Nielsen euristikos\cite{NielsenHeuristicsEn}. Lentelėje bus išskiriamos 10 euristikų. Kiekvienai euristikai priskiriamas skaičius nuo 0 iki 3 reiškiantis defekto sunkumą, kur 0 - defekto nėra, 1 - smulkus ar kosmetinis defektas, 2 - vidutinio sunkumo defektas, 3 - kritinis defektas. Paskutinis stulpelis nusako, kur ir koks defektas buvo pastebėtas (\ref{euristikųlentelėpvz} pav.).
\begin{center}
\begin{tabular}{ |c|c|c| } 
 \hline
	Euristika & Defekto sunkumas & Komentaras \\ \hline
	1) Būsenos matomumas &  &  \\ \hline
	2) Atitikimas realiam pasauliui  &  &  \\ \hline
	3) Naudotojo valdomas dialogas &  &  \\ \hline
	4) Darna ir standartai &  &  \\ \hline
	5) Klaidų prevencija &  &  \\ \hline
	6) Atpažinimas geriau nei atsiminimas &  &  \\ \hline
	7) Naudojimo lankstumas ir efektyvumas &  &  \\ \hline
	8) Estetiškas ir minimalistinis dizainas &  &  \\ \hline
	9) Remti klaidų atpažinimą, jų priežasčių nustatymą ir taisymą &  &  \\ \hline
	10) Parama ir dokumentacija &  &  \\ \hline
\end{tabular}
\captionof{table}{Euristinio vertinimo lentelės pavyzdys}\label{euristikųlentelėpvz}
\end{center}

\section{Puslapio panaudojamumo analizė}
Norint išsiaiškinti dabartinės sistemos panaudojamumo būseną buvo atliktas panaudojamumo testavimas, kurio metu stebima per kiek laiko ir paspaudimų naudotojas pasiekia tikslą. Po testavimo naudotojas buvo apklausiamas kaip jam sekėsi naudotis sistema.



\vspace{0,5cm}
Atlikus puslapio pažintinę peržvalgą rasti šie panaudojamumo trūkumai:
\begin{enumerate}
	\item Navigacijos (\ref{img:registracija} pav.)
	\renewcommand*{\theenumii}{\theenumi.\arabic{enumii}}
	\renewcommand{\labelenumii}{\theenumii}
	\begin{enumerate}
		\item Pasirinkta kategorija mažo kontrasto, taigi sunku pastebėti esamą kategoriją
		\item Išskleistos subkategorijos mažai išsiskiria, taigu sunku atskirti kategorijas nuo subkategorijų
		\item Naviguojant į „Pacientams“, tada „Aktuali informacija“ ir galiausiai „Kaip užsiregistruoti“ pasiekiamas tuščias puslapis. Vietoj to reikia eiti į „Pacientams“ ir „Kaip mus rasti“ arba „e.Paslaugos“ ir „Išankstinė registracija“
		\item Ant mobilių įrenginių navigacijos juosta beveik neįskaitoma, mygtukai per maži (\ref{img:santamobile} pav.)
	\end{enumerate}
	\item Paieškos (\ref{img:paieškarez} pav.)
	\renewcommand*{\theenumii}{\theenumi.\arabic{enumii}}
	\renewcommand{\labelenumii}{\theenumii}
	\begin{enumerate}
		\item Paieškos laukas turi 20 simbolių limitą, kas neleidžia įvesti ilgesnių paieškos užklausų
		\item Vedant simbolius į paiešką nepasiūlomi užklausos variantai
		\item Nėra filtravimo pagal skirtingas kategorijas, pavyzdžiui naujienas atskirti nuo puslapių apie ligoninės skyrius
		\item Nėra filtravimo pagal kalbą
		\item Ant mobilių įrenginių sunku paspausti ant paieškos elementų (\ref{img:santamobile} pav.)
	\end{enumerate}
	\item Rezultatų (\ref{img:paieškarez} pav.)
	\renewcommand*{\theenumii}{\theenumi.\arabic{enumii}}
	\renewcommand{\labelenumii}{\theenumii}
	\begin{enumerate}
		\item Rezultatai neturi nuorodos į kategorijas iš kurių jie kilę
		\item Rezultatų lange viskas pateikta teksto pavidalu, nėra grafinių elementų padedančių skirstyti rezultatus pagal kategorijas
		\item Ant mobilių įrenginių prastai matosi rezultato detalės kaip data ir kategorija (\ref{img:santamobile} pav.)
	\end{enumerate}
\end{enumerate}
\vspace{0,5cm}

Iš peržvalgos matoma trūkumų visose kategorijose, tačiau didžiausias iš jų yra, kad puslapis nepritaikytas mobiliems įrenginiams. Tai yra didelė problema, nes mobilieji įrenginiai vis plačiau naudojami \cite{EmergingmHealthEn} ir šių įrenginių vartotojai turėtų galėti laisvai naudotis puslapiu. Galutinis prototipas turi išspręsti šiuos panaudojomumo trūkumus.
%Medžiagos darbo tema dėstymo skyriuose pateikiamos nagrinėjamos temos detalės:
%pradinė medžiaga, jos analizės ir apdorojimo metodai, sprendimų įgyvendinimas,
%gautų rezultatų apibendrinimas. Šios dalies turinys labai priklauso nuo darbo
%temos. Skyriai gali turėti poskyrius ir smulkesnes sudėtines dalis, kaip
%punktus ir papunkčius.

%Medžiaga turi būti dėstoma aiškiai, pateikiant argumentus. Tekstas dėstomas
%trečiuoju asmeniu, t.y. rašoma ne „aš manau“, bet „autorius mano“, „autoriaus
%nuomone“. Reikėtų vengti informacijos nesuteikiančių frazių, pvz., „...kaip jau
%buvo minėta...“, „...kaip visiems žinoma...“ ir pan., vengti grožinės literatūros
%ar publicistinio stiliaus, gausių metaforų ar panašių meninės išraiškos
%priemonių.

\section{Sprendimo variantai}

\subsection{Alternatyvieji sprendimai}
Bandant sukurti sprendimo maketus bus remiamasi jau egzistuojančiais ligoninių puslapiais, kurie atitinka prisitaikančio dizaino (angl. responsive design) principus. Prisitaikantis dizainas leidžia turėti vieną puslapį, kuris prisitaiko prie įvarių ekrano formų ir dydžių\cite{RWDEn}. Pavyzdiniai puslapiai parinkti iš didmiesčių ligoninių, Vilniaus, Kauno ir Niujorko. Autoriaus subjektyvia nuomone Kauno ligonės puslapis yra ypač geras pavyzdys. Sumuštinio meniu, registracijos mygtukas ir paieška yra geriausiai matomoje vietoje, naudojami dideli mygtukai su aiškiais užrašais bei kontrastingos spalvos, taigi net žmonėms su prastu regėjimu turėtų būti patogu naudotis (\ref{img:kaunomobile} pav.).

Sprendimo maketai bus kuriami naudojant Balsamiq programinę įrangą, nes ji leidžia greitai sukurti grubų maketą ir autoriui jau tekę ja naudotis. Galutinio sprendimo maketas bus kuriamas su Axure RP 9, nes tai leis sukurti maketą, kuris panašesnis į galutinį rezultatą.

\subsection{Sprendimo maketai}
\subsubsection{Pirmas maketas}

\subsubsection{Antras maketas}
\subsubsection{Trečias maketas}

\subsection{Maketų palyginimas}
\subsection{Galutinio sprendimo prototipas}

\section{Reikalavimai galutinio sprendimo įgyvendinimui}
\section{Technologijos galutinio sprendimo įgyvendinimui}

\sectionnonum{Rezultatai ir išvados}

%Rezultatų ir išvadų dalyje turi būti aiškiai išdėstomi pagrindiniai darbo
%rezultatai (kažkas išanalizuota, kažkas sukurta, kažkas įdiegta) ir pateikiamos
%išvados (daromi nagrinėtų problemų sprendimo metodų palyginimai, teikiamos
%rekomendacijos, akcentuojamos naujovės).


%% PAKEISTAS PAVADINIMAS Į 'Šaltiniai'
\printbibliography[heading=bibintoc, title=Šaltiniai]  % Šaltinių sąraše nurodoma panaudota
% literatūra, kitokie šaltiniai. Abėcėlės tvarka išdėstomi darbe panaudotų
% (cituotų, perfrazuotų ar bent paminėtų) mokslo leidinių, kitokių publikacijų
% bibliografiniai aprašai.  Šaltinių sąrašas spausdinamas iš naujo puslapio.
% Aprašai pateikiami netransliteruoti. Šaltinių sąraše negali būti tokių
% šaltinių, kurie nebuvo paminėti tekste.

% \sectionnonum{Sąvokų apibrėžimai}
%\sectionnonum{Santrumpos}
%Sąvokų apibrėžimai ir santrumpų sąrašas sudaromas tada, kai darbo tekste
%vartojami specialūs paaiškinimo reikalaujantys terminai ir rečiau sutinkamos
%santrumpos.

\appendix  % Priedai
% Prieduose gali būti pateikiama pagalbinė, ypač darbo autoriaus savarankiškai
% parengta, medžiaga. Savarankiški priedai gali būti pateikiami ir
% kompaktiniame diske. Priedai taip pat numeruojami ir vadinami. Darbo tekstas
% su priedais susiejamas nuorodomis.

\section{Dabartinio puslapio grafinė vartotojo sąsaja}
\begin{figure}[H]
    \centering
    \includegraphics[scale=0.6]{img/RegistracijosAklavietė}
    \caption{Registracijos aklavietė}
    \label{img:registracija}
\end{figure}

\begin{figure}[H]
    \centering
    \includegraphics[scale=0.6]{img/PaieškosRezultatai}
    \caption{Paieška ir rezultatai}
    \label{img:paieškarez}
\end{figure}

\section{Puslapių atvaizdavimas ant mobilaus įrenginio}
\begin{figure}[H]
    \centering
    \begin{minipage}{.5\textwidth}
    	\centering
    	\includegraphics[scale=0.12]{img/SantaMobile}
    	\caption{Santa.lt ant mobilaus įrenginio}
    	\label{img:santamobile}
    \end{minipage}%
    \begin{minipage}{.5\textwidth}
    	\centering
    	\includegraphics[scale=0.12]{img/VmklMobile}
    	\caption{Vmkl.lt ant mobilaus įrenginio}
    	\label{img:vmklmobile}
    \end{minipage}
\end{figure}

\begin{figure}[H]
    \centering
    \begin{minipage}{.5\textwidth}
    \centering
    \includegraphics[scale=0.12]{img/KaunoMobile}
    \caption{kaunoligonine.lt ant mobilaus įrenginio}
    \label{img:kaunomobile}
    \end{minipage}%
\begin{minipage}{.5\textwidth}
    \centering
    \includegraphics[scale=0.12]{img/NypMobile}
    \caption{nyp.org ant mobilaus įrenginio}
    \label{img:nypmobile}
    \end{minipage}
\end{figure}


%\section{Eksperimentinio palyginimo rezultatai}
% tablesgenerator.com - converts calculators (e.g. excel) tables to LaTeX
%\begin{table}[H]\footnotesize
%  \centering
%  \caption{Lentelės pavyzdys}
%  {\begin{tabular}{|l|c|c|} \hline
%    Algoritmas & $\bar{x}$ & $\sigma^{2}$ \\
%    \hline
%    Algoritmas A  & 1.6335    & 0.5584       \\
%    Algoritmas B  & 1.7395    & 0.5647       \\
%    \hline
%  \end{tabular}}
%  \label{tab:table example}
%\end{table}

\end{document}
